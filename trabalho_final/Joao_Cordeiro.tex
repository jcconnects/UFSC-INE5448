% ===============================
% Template de Relatório de PoC (INE5448)
% IA & Segurança — LabREC/UFSC
% Versão 2.1 - ESTRUTURA VAZIA
% ===============================
\documentclass[12pt,a4paper]{article}

% ---- Codificação, idioma e fontes ----
\usepackage[utf8]{inputenc}
\DeclareUnicodeCharacter{2192}{\ensuremath{\rightarrow}}
\DeclareUnicodeCharacter{2190}{\ensuremath{\leftarrow}}
\DeclareUnicodeCharacter{2194}{\ensuremath{\leftrightarrow}}
\DeclareUnicodeCharacter{00D7}{\ensuremath{\times}}
\DeclareUnicodeCharacter{2265}{\ensuremath{\geq}}
\DeclareUnicodeCharacter{2264}{\ensuremath{\leq}}
\DeclareUnicodeCharacter{00B1}{\ensuremath{\pm}}
\usepackage[T1]{fontenc}
\usepackage[brazilian]{babel}
\usepackage[a4paper,margin=2.5cm]{geometry}
\usepackage{times}

% ---- Matemática, figuras e links ----
\usepackage{amsmath,amssymb}
\usepackage{graphicx}
\usepackage{float}
\usepackage{hyperref}
\usepackage{enumitem}
\usepackage{xcolor}
\usepackage{booktabs}
\usepackage{csquotes}
\usepackage{listings}
\usepackage{subcaption}
\usepackage{multirow}

% ---- Configuração de hyperlinks ----
\hypersetup{
    colorlinks=true,
    linkcolor=blue,
    filecolor=magenta,
    urlcolor=cyan,
    citecolor=blue,
    pdftitle={Relatório de PoC - INE5448},
    pdfpagemode=FullScreen,
}

% ---- Configuração de listagens de código ----
\lstset{
    basicstyle=\ttfamily\small,
    breaklines=true,
    frame=single,
    numbers=left,
    numberstyle=\tiny,
    tabsize=2,
    commentstyle=\color{gray},
    keywordstyle=\color{blue},
    stringstyle=\color{red},
    literate=
        {á}{{\'a}}1 {é}{{\'e}}1 {í}{{\'i}}1 {ó}{{\'o}}1 {ú}{{\'u}}1
        {Á}{{\'A}}1 {É}{{\'E}}1 {Í}{{\'I}}1 {Ó}{{\'O}}1 {Ú}{{\'U}}1
        {à}{{\`a}}1 {è}{{\`e}}1 {ì}{{\`i}}1 {ò}{{\`o}}1 {ù}{{\`u}}1
        {À}{{\`A}}1 {È}{{\'E}}1 {Ì}{{\`I}}1 {Ò}{{\`O}}1 {Ù}{{\`U}}1
        {ä}{{\"a}}1 {ë}{{\"e}}1 {ï}{{\"i}}1 {ö}{{\"o}}1 {ü}{{\"u}}1
        {Ä}{{\"A}}1 {Ë}{{\"E}}1 {Ï}{{\"I}}1 {Ö}{{\"O}}1 {Ü}{{\"U}}1
        {â}{{\^a}}1 {ê}{{\^e}}1 {î}{{\^i}}1 {ô}{{\^o}}1 {û}{{\^u}}1
        {Â}{{\^A}}1 {Ê}{{\^E}}1 {Î}{{\^I}}1 {Ô}{{\^O}}1 {Û}{{\^U}}1
        {ã}{{\~a}}1 {õ}{{\~o}}1 {Ã}{{\~A}}1 {Õ}{{\~O}}1
        {ç}{{\c{c}}}1 {Ç}{{\c{C}}}1
        {ñ}{{\~n}}1 {Ñ}{{\~N}}1
}

\lstdefinelanguage{yaml}{
  keywords={true,false,null,y,n},
  keywordstyle=\color{blue}\bfseries,
  basicstyle=\ttfamily\small,
  sensitive=false,
  comment=[l]{\#},
  morecomment=[s]{/*}{*/},
  commentstyle=\color{gray}\ttfamily,
  stringstyle=\color{red}\ttfamily,
  moredelim=[l][\color{orange}]{\&},
  moredelim=[l][\color{magenta}]{*},
  moredelim=**[il][\color{green}]{:},
  morestring=[b]',
  morestring=[b]",
  literate=
    {á}{{\'a}}1 {é}{{\'e}}1 {í}{{\'i}}1 {ó}{{\'o}}1 {ú}{{\'u}}1
    {â}{{\^a}}1 {ê}{{\^e}}1 {î}{{\^i}}1 {ô}{{\^o}}1 {û}{{\^u}}1
    {ã}{{\~a}}1 {õ}{{\~o}}1 {ç}{{\c{c}}}1 {Ç}{{\c{C}}}1
}

\usepackage[table]{xcolor}
\usepackage{booktabs}

% ---- Bibliografia numérica (biber) ----
\usepackage[backend=biber,style=numeric,sorting=nyt,maxbibnames=99]{biblatex}
\addbibresource{Joao_Cordeiro.bib}

% ---- Informações do documento ----
% INSTRUÇÕES: Substitua os campos abaixo com suas informações
\title{\textbf{SAST para detecção de vulnerabilidades em workflows do n8n}}
\author{João Pedro Schmidt Cordeiro \and Gustavo Zambonin}

\date{\small \today}

% ==== Início do Documento ====
\begin{document}

\maketitle

% ---- Resumo (Abstract) ----
\begin{abstract}
A crescente adoção de plataformas Low-Code/No-Code (LCNC), como o n8n, tem transformado o desenvolvimento de automações empresariais, democratizando a criação de workflows através de interfaces visuais. Entretanto, essa mudança de paradigma introduz uma lacuna crítica de governança: enquanto a plataforma subjacente é protegida por práticas robustas de segurança, os workflows criados pelos usuários, armazenados como arquivos JSON declarativos, não são submetidos ao mesmo rigor de análise de segurança. Este trabalho propõe o desenvolvimento de uma ferramenta de Análise Estática de Segurança de Aplicações (SAST) específica para workflows do n8n, capaz de detectar automaticamente vulnerabilidades como SQL Injection, Command Injection, SSRF, exposição de credenciais, manuseio inseguro de dados e negação de serviço. A metodologia adotada baseia-se em uma estratégia híbrida que combina duas ferramentas do estado da arte: o Agentic Radar, focado em riscos específicos de agentes de IA, e o Semgrep, configurado com regras customizadas para vulnerabilidades tradicionais de aplicações web. Além disso, propõem-se abordagens inovadoras para a criação de regras para o Semgrep, utilizando inteligência artificial e workflows com problemas conhecidos como base para o desenvolvimento e validação de padrões de detecção. A análise detalhada revelou que o Agentic Radar oferece cobertura efetiva de 16,7\% das classes de vulnerabilidades identificadas, enquanto o Semgrep apresenta potencial de cobertura de 66,7\% quando adequadamente configurado. Os resultados demonstram a viabilidade técnica da abordagem proposta, validando que a análise estática de estruturas declarativas JSON não apenas é possível, mas também eficaz para identificação de padrões inseguros. A solução contribui para elevar o nível de segurança das automações desenvolvidas em organizações brasileiras, auxiliando na conformidade com regulamentações como a LGPD e reduzindo riscos de vazamento de dados e ataques cibernéticos em ambientes de desenvolvimento democratizado.
\end{abstract}

\textbf{Palavras-chave:} SAST, n8n, Low-Code/No-Code, Segurança de Aplicações, Análise Estática, Vulnerabilidades, Automação de Workflows, Inteligência Artificial, LGPD.

\tableofcontents

% ==== Seção 1: Introdução ====
\section{Introdução}
\label{sec:introducao}

A crescente adoção de plataformas Low-Code/No-Code (LCNC) representa uma transformação fundamental no desenvolvimento de aplicações e automações empresariais. O n8n, uma proeminente plataforma de automação de workflows de código aberto, exemplifica essa mudança ao capacitar equipes técnicas a conectar APIs, bancos de dados e serviços através de um editor visual intuitivo. Esta democratização do desenvolvimento, embora acelere drasticamente a criação de automações, introduz desafios críticos de segurança que este trabalho busca endereçar.

A mudança de um modelo de codificação imperativa tradicional para um modelo de configuração declarativa, onde a lógica é definida visualmente e armazenada como objetos JSON, introduz um novo paradigma para a segurança de aplicações. Existe uma lacuna crítica de governança: o "código-fonte" da aplicação (arquivo JSON do workflow) não está sujeito ao mesmo rigor de segurança que a plataforma na qual é executado. As equipes de segurança não podem mais depender exclusivamente das garantias de segurança do fornecedor; em vez disso, devem estabelecer seus próprios processos de garantia para os ativos criados na plataforma.

\subsection{Motivação e Contexto}

O interesse por este tema surge de uma necessidade prática identificada no ambiente de trabalho, onde a plataforma n8n é utilizada diariamente. Com o crescimento do número de workflows e a expansão do uso da ferramenta para diferentes áreas organizacionais, observou-se a necessidade crítica de averiguar a segurança dos workflows desenvolvidos. Particularmente preocupante é o fato de que muitos usuários que criam workflows não possuem conhecimento técnico aprofundado em segurança, o que pode resultar no desenvolvimento não intencional de vulnerabilidades dentro do sistema.

Esta experiência prática evidencia a lacuna de governança identificada na literatura, onde o "desenvolvedor cidadão" assume responsabilidades de desenvolvimento sem o treinamento formal necessário para identificar riscos de segurança. A própria n8n implementa práticas de segurança robustas em seu código-fonte, incluindo a utilização de Testes de Segurança de Aplicações Estáticas (SAST) como parte de seu pipeline de Integração Contínua/Entrega Contínua (CI/CD). Contudo, essa varredura não se estende, nem poderia se estender, aos workflows criados pelos seus usuários. A responsabilidade pela concepção de workflows seguros é explicitamente delegada ao usuário.

O potencial de impacto no contexto brasileiro é particularmente significativo considerando a crescente digitalização de processos empresariais e a adoção de ferramentas de automação no país. A Lei Geral de Proteção de Dados (LGPD) e outras regulamentações de segurança cibernética criam uma demanda específica por ferramentas que possam garantir a conformidade e segurança de automações empresariais. Uma ferramenta SAST especializada para n8n junto com a rápida criação de regras novas, de acordo com a demanda, utilizando inteligência artificial, pode contribuir significativamente para elevar o nível de segurança das automações desenvolvidas por organizações brasileiras, reduzindo riscos de vazamento de dados e ataques cibernéticos que podem resultar em penalidades regulatórias e danos reputacionais.

\subsection{Problema Abordado}

O principal desafio de segurança no n8n não é uma falha da plataforma em si, mas sim uma falha potencial na implementação do "usuário-desenvolvedor". Os workflows, armazenados como arquivos JSON, podem conter vulnerabilidades críticas que incluem:

\begin{itemize}
    \item \textbf{SQL Injection (SQLi):} Quando entradas controladas pelo usuário são inseridas em consultas SQL sem sanitização adequada, com dados de gatilhos passados diretamente para parâmetros de nós de banco de dados.

    \item \textbf{Command/Code Injection:} Quando dados não confiáveis são usados em comandos executados no servidor através de nós como Execute Command e Code.

    \item \textbf{Proliferação e Encadeamento de Segredos (Secret Sprawl \& Chaining):} A dispersão de segredos (chaves de API, tokens) em configurações ou logs, com possíveis vazamentos em respostas de webhooks.

    \item \textbf{Falsificação de Solicitação do Lado do Servidor (SSRF):} Quando entradas controladas pelo usuário determinam URLs em nós HTTP Request sem validação adequada.

    \item \textbf{Manuseio Inseguro de Dados:} Falhas na proteção de dados sensíveis durante armazenamento ou transmissão, incluindo uso de HTTP sem TLS e armazenamento não criptografado.

    \item \textbf{Negação de Serviço (DoS):} Workflows suscetíveis a loops infinitos ou operações intensivas, incluindo uso de nós vulneráveis a CVEs conhecidos.
\end{itemize}

A natureza declarativa do JSON do n8n, embora abstraia a complexidade do código tradicional, paradoxalmente torna certos tipos de análise estática mais fáceis e precisas. Ele declara explicitamente as relações de fluxo de dados através do array \texttt{connections}, permitindo regras de análise de contaminação (taint analysis) com elevado grau de confiança e baixa taxa de falsos positivos.

\subsection{Delimitação do Escopo}

Este trabalho foca especificamente no desenvolvimento e validação de uma abordagem híbrida para análise estática de segurança de workflows n8n. O escopo está claramente delimitado:

\textbf{Incluído no escopo:}
\begin{itemize}
    \item Análise estática de arquivos JSON de workflows n8n
    \item Avaliação detalhada de duas ferramentas do estado da arte: Agentic Radar e Semgrep
    \item Detecção das seis classes principais de vulnerabilidades identificadas
    \item Desenvolvimento de metodologia para criação de regras Semgrep utilizando inteligência artificial
    \item Uso de workflows com vulnerabilidades conhecidas como dataset de validação
    \item Análise de viabilidade técnica e métricas de performance
\end{itemize}

\textbf{Excluído do escopo:}
\begin{itemize}
    \item Implementação completa de uma ferramenta SAST de produção
    \item Interface gráfica completa para usuários finais
    \item Análise dinâmica ou execução real de workflows
    \item Análise de nós customizados da comunidade
    \item Correção automática de vulnerabilidades detectadas
    \item Otimização de performance de workflows
    \item Deploy em ambientes de produção corporativos
    \item Análise de vulnerabilidades em nível de infraestrutura da plataforma n8n
\end{itemize}

\subsection{Contribuições Esperadas}

Este trabalho oferece múltiplas contribuições técnicas, metodológicas e científicas para a área de segurança em plataformas LCNC:

\begin{enumerate}
    \item \textbf{Análise Comparativa do Estado da Arte:} Avaliação detalhada e sistemática do Agentic Radar e Semgrep, documentando cobertura de vulnerabilidades, limitações, e aplicabilidade ao contexto n8n, com métricas quantitativas de cobertura (16,7\% e 66,7\% respectivamente).

    \item \textbf{Validação de Viabilidade Técnica:} Demonstração empírica de que análise estática de estruturas declarativas JSON não apenas é possível, mas também eficaz para identificação de padrões inseguros em workflows, validando a premissa fundamental do projeto.

    \item \textbf{Abordagem Híbrida Inovadora:} Proposta de arquitetura que combina ferramentas especializadas (Agentic Radar para riscos de agentes de IA) com ferramentas genéricas configuráveis (Semgrep para vulnerabilidades tradicionais), maximizando cobertura de segurança.

    \item \textbf{Metodologia para Criação de Regras com IA:} Desenvolvimento de abordagem inovadora que utiliza inteligência artificial para acelerar e sistematizar a criação de regras de detecção para o Semgrep, reduzindo o overhead de desenvolvimento.

    \item \textbf{Documentação de Lacunas:} Identificação clara de limitações das ferramentas existentes e oportunidades para desenvolvimento futuro, orientando pesquisas subsequentes na área.
\end{enumerate}

A relevância para a formação profissional está diretamente alinhada com as tendências emergentes do mercado de tecnologia. O desenvolvimento de competências em análise de segurança para plataformas LCNC representa uma especialização altamente demandada no mercado, posicionando o profissional na interseção entre desenvolvimento de baixo código e segurança cibernética. A capacidade de identificar e mitigar riscos em ambientes de desenvolvimento democratizado torna-se um diferencial competitivo crucial à medida que mais organizações adotam estratégias de desenvolvimento cidadão.

% ==== Seção 2: Fundamentação Teórica ====
\section{Fundamentação Teórica e Estado da Arte}
\label{sec:fundamentacao}

Esta seção apresenta a revisão da literatura relevante e análise crítica das ferramentas existentes para análise estática de segurança em plataformas Low-Code/No-Code, com foco específico na plataforma n8n. A análise concentra-se em duas ferramentas principais que representam abordagens distintas ao problema: o Agentic Radar, uma solução moderna baseada em agentes de IA, e o Semgrep, um motor de análise estática clássico e extensível.

\subsection{Técnicas de SAST para Plataformas Low-Code/No-Code}

O cenário de ferramentas para análise de segurança estática de workflows n8n é emergente e fragmentado, apresentando abordagens distintas que podem ser categorizadas em ferramentas dedicadas e adaptáveis. É crucial ressaltar que, até o momento, \textbf{não existe uma abordagem oficial ou solução consolidada para detecção de vulnerabilidades especificamente em workflows do n8n}. Esta lacuna representa tanto um desafio quanto uma oportunidade significativa para pesquisa e desenvolvimento na área.

\subsubsection{Abordagem Moderna: Agentic Radar e IA Agêntica}

O Agentic Radar, desenvolvido pela SPLX AI, emerge como a ferramenta de código aberto mais proeminente especificamente projetada para análise de segurança de workflows em plataformas de automação, incluindo o n8n \cite{splxai_n8n_scanning}. A ferramenta representa uma abordagem moderna ao problema, incorporando capacidades de Inteligência Artificial Agêntica (Agentic AI) para análise de fluxos de trabalho onde agentes de IA interagem com ferramentas, APIs e serviços externos.

A utilização de IA no Agentic Radar manifesta-se em múltiplas dimensões. Primeiro, a ferramenta implementa análise estática consciente do contexto, capaz de interpretar a estrutura JSON de workflows e construir representações gráficas dos fluxos de dados. Segundo, o modo de teste (\texttt{test}) utiliza modelos de linguagem (especificamente a API da OpenAI) para executar testes adversariais em tempo de execução, simulando ataques de injeção de prompt, tentativas de extração de informações sensíveis e geração de conteúdo prejudicial \cite{splxai_medium_scanning}. Esta capacidade de testar dinamicamente a robustez de sistemas de agentes de IA através de entradas adversariais geradas por outros modelos de linguagem representa uma aplicação sofisticada de IA para segurança.

Terceiro, a ferramenta correlaciona padrões identificados nos workflows com categorias de risco estabelecidas pelo OWASP LLM Top 10, sugerindo a utilização de classificação assistida por IA para mapeamento de vulnerabilidades. Quarto, a geração de relatórios com visualizações gráficas interativas e recomendações de remediação contextualizadas indica o uso de técnicas de geração de linguagem natural para produzir explicações técnicas compreensíveis.

O foco atual do Agentic Radar concentra-se em riscos emergentes específicos de sistemas de agentes de IA: injeção de prompt, vazamento de informações de identificação pessoal (PII) através de respostas de modelos de linguagem, geração de conteúdo nocivo e disseminação de desinformação. Esta especialização, embora valiosa para workflows do n8n que integram modelos de linguagem e agentes de IA, não cobre vulnerabilidades tradicionais de aplicações web como SQL Injection, SSRF (Server-Side Request Forgery) ou Command Injection. A análise da cobertura de vulnerabilidades revela que a ferramenta oferece detecção efetiva para aproximadamente 16,7\% das seis classes principais de vulnerabilidades identificadas na literatura para workflows n8n, limitando-se essencialmente a riscos de IA.

\subsubsection{Abordagem Clássica: Semgrep e Análise Baseada em Padrões}

O Semgrep, desenvolvido pela Semgrep Inc., representa uma abordagem clássica de análise estática de código, fundamentada em correspondência de padrões consciente de sintaxe \cite{semgrep_platform}. Diferentemente de ferramentas baseadas em expressões regulares, o Semgrep realiza parsing do código-fonte em árvores sintáticas abstratas (AST), oferecendo compreensão semântica da estrutura do código. A ferramenta suporta mais de 30 linguagens de programação, incluindo análise nativa de formatos estruturados como JSON e YAML, tornando-a particularmente relevante para a análise de workflows do n8n.

A arquitetura do Semgrep fundamenta-se em uma linguagem de regras declarativa, onde padrões de detecção são especificados em arquivos YAML. Esta abordagem permite que equipes de segurança desenvolvam conjuntos de regras personalizados sem a complexidade de manipulação direta de ASTs ou construção de compiladores \cite{semgrep_custom_rules}. No contexto do n8n, regras podem ser desenvolvidas para detectar padrões inseguros específicos: concatenação de entradas não confiáveis em consultas SQL, construção dinâmica de URLs em nós HTTP Request baseada em dados externos, uso de expressões dinâmicas em comandos de execução e exposição de credenciais em configurações.

A capacidade de análise de fluxo de dados (dataflow analysis) do Semgrep é particularmente relevante para a detecção de vulnerabilidades de injeção. A ferramenta permite configurar fontes de dados não confiáveis (sources) — como nós Webhook que recebem entradas externas — e destinos perigosos (sinks) — como parâmetros de consultas SQL ou comandos de execução. O motor de análise então rastreia automaticamente o fluxo de dados contaminados (taint tracking) desde as fontes até os sinks, sinalizando caminhos que não incluem validação ou sanitização adequada \cite{prototype_pollution}.

A análise de cobertura de vulnerabilidades revela que o Semgrep oferece potencial de cobertura para aproximadamente 66,7\% das seis classes principais de vulnerabilidades identificadas (SQL Injection, Command Injection, SSRF e parcialmente Secret Sprawl e Data Handling), com cobertura limitada apenas para detecção de Negação de Serviço (DoS) via análise de ciclos no grafo de execução. Esta capacidade, no entanto, é inteiramente potencial: toda a cobertura depende do desenvolvimento de regras personalizadas específicas para o esquema JSON dos workflows do n8n. Até o momento da presente análise, nenhum conjunto de regras específico para n8n existe no registro comunitário público do Semgrep, representando uma lacuna significativa.

A principal limitação do Semgrep é a ausência de conhecimento nativo do esquema do n8n. A ferramenta não compreende intrinsecamente a semântica dos diferentes tipos de nós (\texttt{n8n-nodes-base.webhook}, \texttt{n8n-nodes-base.mysql}, \texttt{n8n-nodes-base.executeCommand}), a estrutura do array \texttt{connections} que define o grafo de fluxo de dados, ou a sintaxe de expressões dinâmicas \texttt{\{\{ \$json.body.userInput \}\}} que representam o principal vetor de propagação de dados contaminados. Esta ausência exige um investimento significativo no desenvolvimento de regras customizadas, estimado entre 40 e 60 horas de trabalho especializado para alcançar cobertura abrangente.

\subsection{Explicabilidade em Ferramentas de Análise de Segurança}

A explicabilidade refere-se à capacidade de um sistema fornecer insights compreensíveis sobre seu funcionamento e decisões, sendo particularmente crítica em contextos de segurança onde a confiança nas detecções é essencial para a adoção prática das ferramentas \cite{owasp_source_code_analysis}. Tanto o Agentic Radar quanto o Semgrep implementam mecanismos de explicabilidade, embora através de abordagens distintas.

O Agentic Radar gera relatórios em formato HTML com visualizações gráficas interativas dos fluxos de trabalho, identificação clara de vulnerabilidades e explicações técnicas detalhadas sobre por que determinado padrão foi classificado como risco. As recomendações de remediação são contextualizadas e acionáveis, frequentemente incluindo exemplos de código corrigido. O alinhamento explícito com o OWASP LLM Top 10 fornece uma taxonomia padronizada que facilita a comunicação de riscos com stakeholders não técnicos. A capacidade de visualização gráfica é particularmente valiosa para workflows complexos, permitindo que analistas de segurança compreendam o contexto das vulnerabilidades dentro do fluxo de dados completo.

O Semgrep, por sua vez, oferece explicabilidade através da transparência das regras. Cada regra é especificada em YAML legível, onde o padrão de detecção, as condições lógicas e os metadados (severidade, mensagens explicativas, sugestões de correção) são explicitamente declarados. Esta transparência permite que desenvolvedores e analistas de segurança compreendam exatamente o que está sendo detectado e por quê. Os relatórios gerados incluem o padrão correspondido, a localização exata no código (número de linha) e a mensagem explicativa definida na regra. O suporte a metadados arbitrários permite enriquecer as detecções com mapeamentos a frameworks de segurança (OWASP LCNC Top 10, CWE), referências a documentação e exemplos de mitigação.

A principal diferença está na natureza da explicabilidade: o Agentic Radar oferece explicações contextuais baseadas na análise holística do fluxo de trabalho, enquanto o Semgrep oferece explicações baseadas em regras que detalham por que um padrão específico é considerado inseguro. Ambas as abordagens são complementares e valiosas em diferentes contextos de uso.

\subsection{Segurança e Privacidade em Sistemas de Análise Baseados em IA}

A integração de Inteligência Artificial em ferramentas de análise de segurança introduz considerações específicas relacionadas à segurança e privacidade que devem ser cuidadosamente avaliadas, especialmente em contextos corporativos com requisitos rigorosos de conformidade regulatória \cite{n8n_privacy}.

O Agentic Radar, particularmente em seu modo de teste dinâmico, depende da API da OpenAI para execução de testes adversariais. Esta dependência cria múltiplas preocupações. Primeiro, a necessidade de enviar workflows para serviços externos de terceiros levanta questões de privacidade e confidencialidade: workflows corporativos frequentemente contêm lógica de negócio proprietária, credenciais de acesso a sistemas internos e referências a infraestrutura sensível. A transmissão destes artefatos para serviços externos pode violar políticas de segurança organizacionais ou requisitos de conformidade como SOC 2, ISO 27001 ou a Lei Geral de Proteção de Dados (LGPD) no contexto brasileiro \cite{n8n_legal}.

Segundo, a dependência de serviços externos cria questões de disponibilidade e custo operacional. Interrupções na disponibilidade da API da OpenAI impactam diretamente a capacidade de executar análises de segurança, um cenário inaceitável para pipelines de CI/CD críticos onde feedback rápido é essencial. Os custos por chamada de API, embora individualmente pequenos, podem acumular-se significativamente em ambientes corporativos analisando centenas ou milhares de workflows regularmente.

Terceiro, existe a preocupação de segurança da cadeia de suprimentos: a confiança na robustez e segurança dos modelos de linguagem utilizados pela API externa. Vulnerabilidades ou comportamentos inesperados nos modelos podem impactar a confiabilidade das análises, potencialmente produzindo falsos positivos ou, mais gravemente, falsos negativos que deixam vulnerabilidades reais não detectadas.

O Semgrep, como ferramenta de análise estática tradicional, pode operar completamente offline e auto-hospedado, não requerendo transmissão de código para serviços externos. A versão open-source é inteiramente local, endereçando as preocupações de privacidade. No entanto, a plataforma Semgrep Cloud opcional, que oferece dashboards de equipe, agregação de resultados históricos e gerenciamento centralizado de políticas, requer envio de código-fonte para serviços externos operados pela Semgrep Inc. Para organizações com requisitos rigorosos de conformidade, esta limitação restringe o uso à versão auto-hospedada, que carece de algumas funcionalidades de colaboração da plataforma cloud \cite{gitlab_sast}.

A oportunidade de utilizar IA para geração assistida de regras Semgrep, discutida na próxima subseção, deve ser implementada com consideração cuidadosa destes aspectos de privacidade. Modelos de linguagem podem ser utilizados para sugerir regras baseadas em descrições de vulnerabilidades ou exemplos de padrões inseguros, mas a geração deve, idealmente, ocorrer localmente ou através de modelos de IA auto-hospedados para evitar vazamento de conhecimento proprietário sobre padrões de vulnerabilidade específicos da organização.

\subsection{Gaps e Oportunidades}

A análise das ferramentas existentes e do estado da arte revela lacunas significativas e oportunidades promissoras para pesquisa e desenvolvimento na área de análise de segurança para workflows do n8n.

\subsubsection{Gap Fundamental: Ausência de Solução Específica para n8n}

Conforme evidenciado pela análise das seções anteriores, \textbf{não existe atualmente uma ferramenta ou abordagem oficial dedicada especificamente à detecção de vulnerabilidades em workflows do n8n}. O Agentic Radar oferece 16,7\% de cobertura focada em riscos de IA, mas não endereça vulnerabilidades tradicionais de aplicações web. O Semgrep oferece 66,7\% de cobertura potencial, mas requer desenvolvimento significativo de regras customizadas que não existem atualmente no registro público. A lacuna resultante — aproximadamente 83\% das necessidades de segurança não cobertas por soluções prontas — representa um risco substancial para organizações que dependem da plataforma n8n para automações críticas de negócio.

Esta lacuna é particularmente preocupante considerando a mudança de paradigma de segurança identificada na literatura: o workflow JSON não é meramente configuração, mas sim o código-fonte da aplicação, e deve ser submetido ao mesmo rigor de análise que código tradicional \cite{owasp_lcnc_top10}. A ausência de ferramentas adequadas resulta na perpetuação da lacuna de governança, onde workflows potencialmente inseguros são implantados em produção sem análise automatizada.

\subsubsection{Oportunidade: IA para Geração Assistida de Regras Semgrep}

Uma oportunidade particularmente promissora identificada nesta análise é a utilização de Inteligência Artificial para geração e otimização assistida de regras de detecção Semgrep específicas para o esquema do n8n. O desenvolvimento manual de regras, estimado em 40 a 60 horas de trabalho especializado, representa uma barreira significativa para a adoção. O modelo de linguagem de grande escala Claude 3.5 Sonnet pode ser aplicado para:

\begin{enumerate}
  \item \textbf{Geração de regras a partir de descrições em linguagem natural:} Analistas de segurança poderiam descrever vulnerabilidades em linguagem natural (ex.: "detectar concatenação de entrada de webhook em consulta SQL sem sanitização") e o modelo geraria a regra Semgrep correspondente em YAML, incluindo padrões sintáticos, condições lógicas e metadados apropriados.

  \item \textbf{Otimização de regras existentes para redução de falsos positivos:} Modelos de IA poderiam analisar regras que produzem taxas elevadas de falsos positivos e sugerir refinamentos nas condições lógicas, adicionando verificações contextuais que distinguem padrões genuinamente inseguros de uso legítimo.

  \item \textbf{Aprendizado de padrões a partir de workflows reais:} Análise de datasets de workflows (como os 2.000+ workflows públicos identificados na literatura \cite{n8n_workflow_analysis}) através de técnicas de aprendizado não supervisionado poderia identificar clusters de padrões comuns, distinguindo práticas seguras de inseguras e gerando regras que detectam desvios de padrões seguros estabelecidos.

  \item \textbf{Manutenção automatizada de regras:} À medida que a plataforma n8n evolui, adicionando novos tipos de nós ou modificando estruturas de parâmetros, modelos de IA poderiam analisar changelogs e documentação de API para sugerir atualizações nas regras existentes ou identificar necessidade de novas regras para cobrir funcionalidades introduzidas.
\end{enumerate}

Esta abordagem híbrida — motor de análise clássico (Semgrep) potencializado por IA generativa para desenvolvimento de conhecimento de segurança — combina as forças de ambos os paradigmas: a confiabilidade, transparência e performance da análise estática baseada em padrões com a flexibilidade, capacidade de adaptação e eficiência de desenvolvimento proporcionadas por modelos de linguagem.

A implementação desta abordagem utiliza prompts especializados e few-shot learning, fornecendo ao Claude 3.5 Sonnet exemplos de regras Semgrep existentes para outras plataformas e solicitando adaptação para o esquema JSON do n8n. A validação das regras geradas por IA através de testes automatizados em workflows é essencial para garantir confiabilidade antes da implantação em ambientes de produção.

\subsubsection{Oportunidade: Arquitetura Híbrida para Cobertura Completa}

A análise comparativa das ferramentas sugere que uma arquitetura híbrida, combinando o Agentic Radar para riscos de IA (16,7\% de cobertura) com Semgrep customizado para vulnerabilidades tradicionais (66,7\% de cobertura potencial), poderia alcançar cobertura próxima a 100\% das seis classes principais de vulnerabilidades identificadas. Esta estratégia maximiza as forças de ambas as abordagens enquanto mitiga suas fraquezas individuais: o Agentic Radar oferece análise especializada em agentes de IA com compreensão nativa de workflows, enquanto o Semgrep oferece motor de análise estática poderoso, flexível e de alta performance para detecção de padrões customizados.

A integração destas ferramentas em um pipeline unificado, com agregação de resultados, classificação por severidade e categoria OWASP LCNC Top 10, e geração de relatórios consolidados, representa uma oportunidade de pesquisa significativa para o desenvolvimento de uma solução SAST verdadeiramente abrangente para o ecossistema n8n.

% ==== Seção 3: Metodologia ====
\section{Metodologia}
\label{sec:metodologia}

Esta seção descreve a abordagem metodológica adotada para o desenvolvimento e validação da ferramenta SAST proposta, detalhando o processo de análise das ferramentas existentes, a metodologia inovadora de geração de regras assistida por inteligência artificial e os procedimentos de validação empregados.

\subsection{Abordagem Metodológica Geral}

A pesquisa adota uma metodologia exploratória e experimental, combinando análise comparativa de ferramentas do estado da arte com desenvolvimento de proof-of-concept. A abordagem metodológica fundamenta-se em três pilares principais:

\begin{enumerate}
    \item \textbf{Pesquisa exploratória:} Análise comparativa detalhada das ferramentas existentes (Agentic Radar e Semgrep), documentando suas capacidades, limitações e aplicabilidade ao contexto específico dos workflows do n8n.

    \item \textbf{Desenvolvimento de abordagem híbrida:} Proposta de arquitetura que combina o Agentic Radar para detecção de vulnerabilidades específicas de agentes de IA com regras customizadas do Semgrep para vulnerabilidades tradicionais de aplicações web, maximizando a cobertura de segurança.

    \item \textbf{Uso de IA para aceleração de desenvolvimento:} Aplicação do modelo de linguagem de grande escala Claude 3.5 Sonnet para geração assistida de regras de detecção Semgrep, reduzindo significativamente o overhead de desenvolvimento manual estimado em 40 a 60 horas.
\end{enumerate}



\subsection{Análise das Ferramentas Existentes}

A análise comparativa das ferramentas do estado da arte seguiu um protocolo estruturado para garantir avaliação objetiva e sistemática das capacidades e limitações de cada solução.

\subsubsection{Critérios de Avaliação}

As ferramentas foram avaliadas segundo quatro critérios principais, estabelecidos com base nos requisitos identificados na revisão da literatura. O primeiro critério avalia a \textbf{cobertura de vulnerabilidades}, medindo a capacidade de detectar as seis classes principais identificadas: SQL Injection, Command/Code Injection, Secret Sprawl \& Chaining, SSRF, Insecure Data Handling e DoS. Para cada classe, a cobertura foi classificada como ALTO, MÉDIO, BAIXO ou NENHUMA, permitindo o cálculo de percentuais quantitativos de cobertura.

O segundo critério analisa a \textbf{performance de análise}, considerando o tempo de execução para análise de workflows, o throughput (medido em workflows processados por hora) e a escalabilidade para repositórios contendo centenas de workflows. O terceiro critério examina a \textbf{facilidade de uso}, avaliando a complexidade de instalação e configuração, a clareza da documentação, a intuitividade da interface de linha de comando e o número de comandos necessários para operação básica.

O quarto critério avalia a \textbf{qualidade dos relatórios} gerados, considerando a clareza das detecções, a classificação de severidade, a profundidade das explicações técnicas, a qualidade das sugestões de remediação, o mapeamento a frameworks de segurança estabelecidos (como o OWASP LCNC Top 10) e as capacidades de visualização oferecidas.

\subsubsection{Processo de Análise}

O processo de análise de cada ferramenta seguiu um protocolo sistemático estruturado em quatro etapas principais. Inicialmente, realizou-se a \textbf{instalação e configuração} das ferramentas Agentic Radar (via \texttt{pip install agentic-radar}) e Semgrep (via \texttt{pip install semgrep}) em ambiente Python 3.10+. Durante esta etapa, foram documentadas todas as dependências, requisitos de configuração (como variáveis de ambiente) e possíveis incompatibilidades encontradas, garantindo a reprodutibilidade do ambiente de testes.

Na segunda etapa, conduziram-se \textbf{testes com workflows de exemplo}, executando cada ferramenta sobre um conjunto de workflows representativos que cobrem diferentes tipos de nós relevantes para segurança, incluindo Webhook, MySQL, HTTP Request, Execute Command e Code. Esta diversidade de nós permitiu avaliar a capacidade das ferramentas em diferentes contextos de vulnerabilidades.

A terceira etapa consistiu na \textbf{análise de capacidades de detecção}, avaliando as seis classes de vulnerabilidades em três workflows de teste e verificando se cada ferramenta detecta os padrões inseguros implementados. Para cada detecção ou ausência de detecção, documentou-se sistematicamente se tratava-se de true positive, false positive, true negative ou false negative, permitindo análise quantitativa posterior.

A etapa final envolveu a \textbf{documentação de cobertura e limitações}, realizando o mapeamento sistemático da cobertura de cada ferramenta para as seis classes de vulnerabilidades, identificando lacunas fundamentais (vulnerabilidades não cobertas) e documentando limitações técnicas, operacionais e de privacidade. Os resultados desta análise comparativa foram consolidados nas seções anteriores deste documento, fornecendo a base empírica para a proposta da abordagem híbrida.

\subsection{Geração de Regras Semgrep Assistida por IA}

A metodologia de geração assistida de regras representa a contribuição metodológica mais inovadora deste trabalho, aplicando técnicas de inteligência artificial generativa para acelerar e sistematizar o desenvolvimento de conhecimento de segurança específico do domínio do n8n.

\subsubsection{Preparação de Workflows de Teste}

O desenvolvimento de regras de detecção efetivas requer workflows de teste que contenham as vulnerabilidades a serem detectadas. Para atender a este requisito, desenvolveram-se três workflows de teste que implementam deliberadamente múltiplas falhas de segurança, cobrindo as seis classes de vulnerabilidades identificadas na literatura. Cada workflow foi cuidadosamente construído para representar cenários realistas onde desenvolvedores inadvertidamente introduzem vulnerabilidades.

Exemplos das vulnerabilidades implementadas incluem SQL Injection através de nó Webhook recebendo entrada do usuário que é passada diretamente via expressão \texttt{\{\{ \$json.body.query \}\}} para o parâmetro \texttt{query} de nó MySQL sem sanitização; Command Injection onde entrada de webhook é utilizada em nó Execute Command via \texttt{\{\{ \$json.body.command \}\}}; e SSRF através de URL dinâmica \texttt{\{\{ \$json.body.url \}\}} em nó HTTP Request sem validação adequada.

Todos os três workflows de teste foram exaustivamente documentados com especificação clara das vulnerabilidades implementadas, incluindo a localização exata dos padrões inseguros (identificando o nó e parâmetro específico), as categorias OWASP LCNC Top 10 correspondentes e as severidades esperadas para cada vulnerabilidade. Esta documentação detalhada serve como ground truth para validação das regras de detecção geradas.

\subsubsection{Metodologia de Geração de Regras com IA}

A geração assistida por inteligência artificial de regras Semgrep segue um processo iterativo estruturado que se inicia com a análise detalhada do esquema JSON dos workflows n8n. Antes da geração de regras propriamente dita, foi necessário compreender profundamente a estrutura JSON dos workflows exportados pelo n8n. Esta análise identificou a estrutura do array \texttt{nodes} e seus campos relevantes (\texttt{type}, \texttt{parameters}, \texttt{name}), a estrutura do array \texttt{connections} que define o grafo de fluxo de dados, a sintaxe de expressões dinâmicas \texttt{\{\{ ... \}\}} e sua semântica, além dos tipos de nós críticos para segurança e seus parâmetros perigosos.

Com este conhecimento do esquema consolidado, utilizou-se o modelo de linguagem de grande escala Claude 3.5 Sonnet através de prompts especializados estruturados em três componentes essenciais. O primeiro componente fornece o \textit{Contexto}, descrevendo o esquema JSON do n8n, a estrutura de workflows e a sintaxe de expressões dinâmicas. O segundo componente especifica o \textit{Objetivo}, detalhando claramente a vulnerabilidade a ser detectada e o comportamento esperado da regra. O terceiro componente define o \textit{Formato}, estabelecendo os requisitos de estrutura da regra Semgrep (YAML), incluindo campos obrigatórios como \texttt{pattern}, \texttt{message}, \texttt{severity} e metadados necessários.

\textbf{Exemplo de prompt estruturado:}

\begin{quote}
\textit{"Você é um especialista em segurança de aplicações e na ferramenta Semgrep. Crie uma regra Semgrep em formato YAML que detecta vulnerabilidades de SQL Injection em workflows do n8n.}

\textit{Contexto: Workflows do n8n são arquivos JSON onde o array 'nodes' contém objetos representando etapas de processamento. Nós de banco de dados MySQL têm type='n8n-nodes-base.mysql' e o parâmetro 'parameters.query' contém a consulta SQL. Expressões dinâmicas usam sintaxe \{\{ \$json.campo \}\}.}

\textit{Objetivo: Detectar quando o parâmetro 'query' de um nó MySQL contém expressões \{\{ ... \}\} que referenciam dados de entrada não confiáveis, indicando possível concatenação de strings em SQL sem sanitização adequada.}

\textit{Formato: A regra deve incluir id, message explicativa, severity='ERROR', metadata com owasp\_lcnc='LCNC-SEC-06' e CWE, e padrões que identifiquem o JSON específico do n8n."}
\end{quote}

O processo de geração seguiu um ciclo iterativo de refinamento contínuo, onde cada regra é progressivamente aprimorada até alcançar os critérios de qualidade estabelecidos. Este ciclo compreende cinco etapas principais:

\begin{enumerate}
    \item \textbf{Geração inicial:} Claude 3.5 Sonnet gera regra Semgrep baseada no prompt
    \item \textbf{Teste:} Execução da regra sobre os três workflows de teste
    \item \textbf{Avaliação:} Verificação de detecções (true/false positives/negatives)
    \item \textbf{Refinamento:} Se necessário, prompt é ajustado com feedback específico ("A regra atual gera falso positivo quando X, ajuste para...") e nova regra é gerada
    \item \textbf{Convergência:} Ciclo repete até regra alcançar precisão e recall satisfatórios
\end{enumerate}

A experiência prática demonstrou que este processo tipicamente convergiu em 2 a 4 iterações por regra, reduzindo drasticamente o tempo de desenvolvimento comparado à abordagem manual. Esta eficiência representa uma das principais contribuições metodológicas do trabalho, tornando viável a criação de conjuntos abrangentes de regras de detecção.

\subsubsection{Validação das Regras Geradas}

Todas as regras geradas por IA foram submetidas a um processo rigoroso de validação antes da incorporação no conjunto de regras final, garantindo sua confiabilidade e precisão. A primeira etapa de validação consistiu na execução das regras sobre os três workflows de teste, com documentação sistemática de todos os resultados. Esta etapa permitiu verificar empiricamente se as regras detectam corretamente as vulnerabilidades implementadas e se geram falsos positivos em contextos seguros.

Quando necessário, realizou-se ajuste manual para redução de falsos positivos através do refinamento das condições lógicas das regras. Este refinamento envolveu a adição de padrões negativos (\texttt{pattern-not}) e contextualizações (\texttt{pattern-inside}) para excluir padrões que, embora sintaticamente similares aos inseguros, representam uso legítimo e seguro da funcionalidade. Esta etapa manual é essencial para garantir que as regras sejam úteis em ambientes de produção, onde taxas elevadas de falsos positivos reduzem drasticamente a confiança dos desenvolvedores na ferramenta.

Finalmente, cada regra validada foi exaustivamente documentada, incluindo descrição técnica da vulnerabilidade detectada, exemplos de código inseguro e seguro, mapeamento para frameworks de segurança estabelecidos (OWASP LCNC Top 10 e CWE), justificativa da severidade atribuída e sugestões concretas de remediação. Esta documentação completa garante que as regras não apenas detectem vulnerabilidades, mas também eduquem os desenvolvedores sobre as melhores práticas de segurança.

\subsection{Validação e Métricas}

O processo de validação da abordagem proposta combina avaliação quantitativa (métricas de detecção) e qualitativa (cobertura de vulnerabilidades e utilidade prática).

\subsubsection{Testes de Detecção}

A validação quantitativa das capacidades de detecção seguiu um protocolo experimental rigorosamente estruturado para garantir resultados objetivos e reproduzíveis. Ambas as ferramentas — Agentic Radar e Semgrep com regras customizadas — foram executadas sobre os três workflows de teste, com documentação sistemática de todos os resultados obtidos. Esta documentação detalhada permite rastreabilidade completa e validação independente dos achados.

Cada detecção ou ausência de detecção foi classificada utilizando a taxonomia padrão de sistemas de detecção: \textit{True Positive (TP)} para vulnerabilidades reais corretamente identificadas, \textit{False Positive (FP)} para detecções em código seguro (falsos alarmes), \textit{True Negative (TN)} para ausência correta de detecção em código seguro, e \textit{False Negative (FN)} para vulnerabilidades reais não detectadas.

A partir desta classificação, calcularam-se as métricas padrão de avaliação de sistemas de detecção:

\begin{itemize}
    \item Precisão: $P = \frac{TP}{TP + FP}$ (proporção de detecções corretas)
    \item Recall: $R = \frac{TP}{TP + FN}$ (proporção de vulnerabilidades encontradas)
    \item F1-Score: $F_1 = 2 \cdot \frac{P \cdot R}{P + R}$ (média harmônica)
    \item Taxa de Falsos Positivos: $\frac{FP}{FP + TN}$
\end{itemize}

Estas métricas foram calculadas tanto por ferramenta individual quanto para a abordagem híbrida combinada, permitindo comparação objetiva das capacidades de detecção e demonstrando quantitativamente os benefícios da estratégia híbrida proposta.

\subsubsection{Avaliação de Cobertura}

A avaliação de cobertura verificou sistematicamente a capacidade da abordagem proposta de endereçar o espectro completo de vulnerabilidades identificadas na revisão da literatura. Para cada uma das seis classes de vulnerabilidades — SQL Injection, Command Injection, Secret Sprawl, SSRF, Insecure Data Handling e DoS — verificou-se se ao menos uma das ferramentas da abordagem híbrida oferece capacidade de detecção efetiva.

A análise comparativa quantitativa revelou diferenças significativas na cobertura oferecida por cada ferramenta individualmente. O Agentic Radar isolado oferece cobertura de 16,7\% (1 de 6 classes), focando especificamente em riscos emergentes de agentes de IA. O Semgrep customizado isolado apresenta cobertura potencial de 66,7\% (4 de 6 classes com cobertura ALTO), endereçando vulnerabilidades tradicionais de aplicações web. A abordagem híbrida, combinando ambas as ferramentas, alcança cobertura próxima a 100\% através da complementaridade estratégica das capacidades de cada solução.

Adicionalmente, documentaram-se as lacunas residuais, identificando categorias de vulnerabilidades que permanecem parcialmente cobertas ou requerem capacidades de análise além do escopo das ferramentas utilizadas. Um exemplo notável é a detecção de Negação de Serviço (DoS) através de análise completa de ciclos no grafo de execução, que requereria capacidades de análise de fluxo de controle mais sofisticadas do que as oferecidas pelas ferramentas atuais.

Os resultados desta validação são apresentados detalhadamente na Seção 7 (Avaliação Experimental), fornecendo evidência empírica robusta da viabilidade e efetividade da abordagem proposta para detecção de vulnerabilidades em workflows do n8n.

% ==== Seção 4: Objetivos e Métricas ====
\section{Objetivos e Métricas de Sucesso}
\label{sec:objetivos}

Esta seção estabelece os objetivos gerais e específicos do trabalho, definindo critérios quantitativos e qualitativos para avaliar o sucesso da validação da abordagem proposta. É importante ressaltar que o foco não é desenvolver uma ferramenta SAST completa de produção, mas sim validar a viabilidade técnica da abordagem híbrida e da metodologia de geração de regras assistida por inteligência artificial.

\subsection{Objetivo Geral}

O objetivo geral deste trabalho é validar a \textbf{viabilidade técnica} de uma abordagem híbrida de análise estática de segurança para workflows da plataforma n8n, combinando o Agentic Radar (para detecção de riscos específicos de agentes de IA) com o Semgrep configurado através de regras customizadas (para vulnerabilidades tradicionais de aplicações web). Especificamente, busca-se:

\begin{enumerate}
    \item Demonstrar empiricamente que a análise estática de estruturas declarativas JSON não apenas é \textbf{possível}, mas também \textbf{efetiva} para identificação de padrões inseguros em workflows, validando a premissa fundamental do projeto.

    \item Desenvolver e validar uma \textbf{metodologia de geração assistida de regras} de detecção para o Semgrep utilizando o modelo de linguagem de grande escala Claude 3.5 Sonnet, demonstrando que inteligência artificial pode acelerar significativamente o desenvolvimento de conhecimento de segurança específico do domínio.

    \item Avaliar comparativamente as capacidades e limitações do Agentic Radar e Semgrep no contexto específico do n8n, documentando suas coberturas de vulnerabilidades, limitações técnicas e aplicabilidade.
\end{enumerate}

O trabalho não visa criar uma ferramenta de produção completa, mas estabelecer a base científica e metodológica que comprove a viabilidade da abordagem, orientando pesquisas futuras e desenvolvimentos práticos na área de segurança para plataformas Low-Code/No-Code.

\subsection{Objetivos Específicos}

Os objetivos específicos do trabalho são organizados em cinco categorias principais, cada uma endereçando um aspecto crítico da validação da abordagem proposta:

\begin{enumerate}[label=\textbf{OE\arabic*:}, left=1.5em]
    \item \textbf{Realizar análise comparativa sistemática do Agentic Radar e Semgrep:} Avaliar detalhadamente ambas as ferramentas do estado da arte quanto à cobertura de vulnerabilidades para cada uma das seis classes identificadas (SQL Injection, Command/Code Injection, Secret Sprawl \& Chaining, SSRF, Insecure Data Handling e DoS), documentando sistematicamente limitações técnicas, operacionais e de privacidade, e analisando a aplicabilidade de cada ferramenta ao contexto específico dos workflows n8n. Esta análise deve produzir métricas quantitativas de cobertura (percentual de classes adequadamente cobertas) e caracterização qualitativa das capacidades de cada solução.

    \item \textbf{Desenvolver metodologia para geração assistida de regras Semgrep:} Criar e documentar uma abordagem sistemática para utilização do Claude 3.5 Sonnet na geração de regras de detecção Semgrep específicas para o esquema JSON dos workflows n8n. A metodologia deve incluir estruturação de prompts especializados, processo iterativo de refinamento de regras, e validação de funcionalidade através de testes com workflows conhecidos. O objetivo é demonstrar que a geração assistida por IA pode reduzir significativamente o overhead de desenvolvimento manual estimado em 40 a 60 horas.

    \item \textbf{Criar conjunto inicial de regras Semgrep customizadas:} Desenvolver regras de detecção para as classes de vulnerabilidades não adequadamente cobertas pelo Agentic Radar, priorizando SQL Injection, Command/Code Injection, SSRF e Secret Sprawl. Cada regra deve incluir padrões de detecção, condições lógicas, metadados apropriados (severidade, categoria OWASP LCNC Top 10, CWE) e mensagens explicativas. As regras devem ser testadas e refinadas iterativamente até demonstrarem capacidade de detectar as vulnerabilidades-alvo nos workflows de teste.

    \item \textbf{Validar a abordagem proposta através de testes empíricos:} Executar ambas as ferramentas (Agentic Radar e Semgrep com regras customizadas) sobre workflows de teste que implementam deliberadamente as seis classes de vulnerabilidades identificadas. Documentar sistematicamente as detecções (true positives, false positives, false negatives), calcular métricas de cobertura da abordagem híbrida e validar que a combinação das ferramentas oferece cobertura significativamente superior a qualquer ferramenta isolada.

    \item \textbf{Documentar lacunas, limitações e oportunidades:} Identificar claramente as limitações da abordagem proposta, classes de vulnerabilidades que permanecem parcialmente cobertas ou requerem capacidades de análise além do escopo das ferramentas utilizadas, e oportunidades para pesquisa futura. Esta documentação deve orientar trabalhos subsequentes na área, estabelecendo um roteiro claro para evolução da solução.
\end{enumerate}

\subsection{Métricas de Avaliação}

A avaliação do sucesso do trabalho será realizada através de critérios objetivos quantitativos e critérios qualitativos que, em conjunto, permitem caracterizar a viabilidade e utilidade da abordagem proposta.

\subsubsection{Métricas Quantitativas}

As métricas quantitativas estabelecem valores-alvo objetivos para aspectos mensuráveis do trabalho, permitindo avaliação empírica do alcance dos objetivos estabelecidos.

\begin{table}[H]
\centering \small
\caption{Métricas quantitativas e valores-alvo}
\label{tab:metricas-quantitativas}
\begin{tabular}{@{}p{4.5cm}p{5.5cm}p{2.5cm}@{}}
\toprule
\textbf{Métrica} & \textbf{Definição} & \textbf{Meta} \\ \midrule
\textbf{Cobertura de Vulnerabilidades (Híbrida)} & Percentual das 6 classes de vulnerabilidades com detecção efetiva através da abordagem híbrida (Agentic Radar + Semgrep) & $\geq$ 80\% \\[0.3cm]
\textbf{Cobertura Agentic Radar} & Percentual das 6 classes adequadamente cobertas pelo Agentic Radar isoladamente & Documentar \% real \\[0.3cm]
\textbf{Cobertura Semgrep} & Percentual das 6 classes cobertas por regras Semgrep customizadas & Documentar \% real \\[0.3cm]
\textbf{Regras Desenvolvidas} & Número de regras Semgrep funcionais criadas utilizando metodologia assistida por IA & $\geq$ 4 regras (uma por classe não coberta) \\[0.3cm]
\textbf{Redução de Tempo} & Economia de tempo no desenvolvimento de regras com IA vs. estimativa de desenvolvimento manual (8-12h por regra) & Documentar economia em horas \\
\bottomrule
\end{tabular}
\end{table}

A métrica de cobertura de vulnerabilidades é calculada como a proporção de classes de vulnerabilidades (das seis identificadas) para as quais existe ao menos uma ferramenta na abordagem híbrida capaz de detectar efetivamente a vulnerabilidade. A meta de 80\% (5 de 6 classes) reconhece que algumas categorias, particularmente DoS através de análise completa de ciclos no grafo de execução, podem requerer capacidades de análise de grafos além do escopo nativo das ferramentas utilizadas.

A métrica de redução de tempo compara o tempo real necessário para desenvolver cada regra Semgrep utilizando Claude 3.5 Sonnet (incluindo tempo de estruturação de prompts, geração, teste e refinamento) com a estimativa de desenvolvimento manual de 8 a 12 horas por regra baseada na literatura de desenvolvimento de regras SAST customizadas. Esta métrica valida a contribuição metodológica do trabalho, demonstrando quantitativamente o valor da geração assistida por inteligência artificial.

\subsubsection{Métricas Qualitativas}

As métricas qualitativas avaliam aspectos não diretamente quantificáveis, mas essenciais para caracterizar a viabilidade prática e utilidade da abordagem proposta. Cada métrica é avaliada através de critérios específicos de aceitação.

\textbf{Viabilidade Técnica:}

A validação fundamental do trabalho é demonstrar que a análise estática de workflows n8n é tecnicamente viável e efetiva. Os critérios de avaliação incluem:

\begin{itemize}
    \item A análise estática de arquivos JSON de workflows n8n permite identificação de padrões inseguros com precisão suficiente para ser útil em ambientes reais?
    \item As regras Semgrep geradas por IA utilizando Claude 3.5 Sonnet são funcionais e detectam corretamente as vulnerabilidades-alvo nos workflows de teste?
    \item A abordagem híbrida fecha as lacunas de cobertura das ferramentas isoladas, oferecendo cobertura complementar efetiva?
    \item As limitações identificadas são documentadas de forma clara e honesta, permitindo avaliação realista da aplicabilidade da solução?
\end{itemize}

\textbf{Usabilidade:}

Embora o foco não seja desenvolver uma ferramenta de produção completa, a viabilidade prática requer que as ferramentas sejam razoavelmente acessíveis:

\begin{itemize}
    \item A instalação e configuração do Agentic Radar e Semgrep é viável com complexidade aceitável (máximo de 5 comandos por ferramenta)?
    \item A documentação disponível (tanto das ferramentas quanto da metodologia proposta) é suficiente para permitir reprodução independente do estudo por outros pesquisadores?
    \item A execução das análises sobre workflows de teste é direta, sem requerer expertise profunda em detalhes de implementação das ferramentas?
\end{itemize}

\textbf{Clareza de Resultados:}

A utilidade prática das detecções depende da qualidade da comunicação dos resultados:

\begin{itemize}
    \item Os relatórios gerados pelas ferramentas identificam claramente as vulnerabilidades detectadas, incluindo localização exata no workflow (nó e parâmetro específico)?
    \item Existe mapeamento explícito das detecções para categorias reconhecidas do OWASP LCNC Top 10, facilitando comunicação com stakeholders?
    \item As explicações técnicas fornecidas são compreensíveis e as sugestões de remediação são acionáveis?
    \item A documentação distingue claramente entre detecções confirmadas (true positives) e potenciais falsos positivos, orientando o analista na interpretação dos resultados?
\end{itemize}

\subsection{Benchmarks Comparativos}

A avaliação da contribuição da abordagem híbrida requer comparação sistemática com três cenários baseline, cada um representando uma estratégia alternativa de análise de segurança para workflows n8n. A Tabela \ref{tab:benchmarks} sintetiza as coberturas esperadas para cada cenário.

\begin{table}[H]
\centering
\caption{Benchmarks comparativos de cobertura de vulnerabilidades}
\label{tab:benchmarks}
\begin{tabular}{@{}p{4cm}p{3cm}p{6.5cm}@{}}
\toprule
\textbf{Cenário} & \textbf{Cobertura Esperada} & \textbf{Observações} \\ \midrule
\textbf{Agentic Radar isolado} & $\sim$16,7\% (1/6 classes) & Baseline 1: Ferramenta especializada com foco em riscos emergentes de agentes de IA (injeção de prompt, vazamento de PII, geração de conteúdo prejudicial). Não cobre vulnerabilidades tradicionais de aplicações web. \\[0.5cm]
\textbf{Semgrep isolado} & $\sim$66,7\% (4/6 classes) & Baseline 2: Potencial teórico para vulnerabilidades tradicionais (SQL Injection, Command Injection, SSRF, Secret Sprawl), mas requer desenvolvimento completo de regras customizadas. Sem regras, cobertura é 0\%. \\[0.5cm]
\textbf{Abordagem Híbrida} & $\sim$83-100\% (5-6/6 classes) & Proposta: Combinação complementar de Agentic Radar (riscos de IA) + Semgrep customizado (vulnerabilidades tradicionais). Maximiza cobertura ao combinar forças de ambas as ferramentas. \\ \bottomrule
\end{tabular}
\end{table}

O cenário \textbf{Agentic Radar isolado} serve como baseline para demonstrar que ferramentas especializadas, embora valiosas para seus domínios específicos (riscos de agentes de IA), são insuficientes como solução autônoma para a segurança abrangente de workflows n8n. A cobertura de aproximadamente 16,7\% (essencialmente limitada a riscos de IA, que representam uma das seis classes quando consideramos workflows que integram modelos de linguagem) estabelece o limite inferior de cobertura.

O cenário \textbf{Semgrep isolado} demonstra o potencial teórico de ferramentas genéricas configuráveis, mas também evidencia a barreira crítica: sem desenvolvimento de regras customizadas, a cobertura é zero. A cobertura potencial de 66,7\% (4 de 6 classes: SQL Injection, Command Injection, SSRF e parcialmente Secret Sprawl) só é alcançada após investimento significativo em desenvolvimento de regras, estimado entre 40 e 60 horas de trabalho especializado. Este cenário estabelece o limite superior para abordagens baseadas exclusivamente em ferramentas tradicionais de análise estática.

O cenário \textbf{Abordagem Híbrida} proposta visa superar as limitações de ambos os baselines através da estratégia complementar: utilizar o Agentic Radar para o que faz bem (riscos de IA) e desenvolver regras Semgrep para preencher as lacunas restantes (vulnerabilidades tradicionais). A meta de 83-100\% de cobertura reconhece que algumas categorias, particularmente DoS através de análise completa de ciclos no grafo, podem permanecer parcialmente cobertas, mas a abordagem deve demonstrar cobertura substancialmente superior a qualquer ferramenta isolada.

A validação empírica da abordagem híbrida envolverá execução de todos os três cenários sobre o mesmo conjunto de workflows de teste, permitindo comparação direta das taxas de detecção e demonstração quantitativa do valor agregado pela estratégia híbrida.

\subsection{Critérios de Sucesso}

O trabalho será considerado bem-sucedido se, e somente se, os seguintes cinco critérios forem satisfeitos:

\begin{enumerate}
    \item \textbf{Viabilidade Técnica Demonstrada:} Ambas as ferramentas (Agentic Radar e Semgrep) conseguem analisar arquivos JSON de workflows n8n e detectar ao menos uma vulnerabilidade cada em workflows de teste. Esta validação fundamental confirma que a análise estática de estruturas declarativas não apenas é teoricamente possível, mas operacionalmente viável com ferramentas existentes.

    \item \textbf{Metodologia com IA Validada:} O Claude 3.5 Sonnet gera regras Semgrep funcionais (que detectam corretamente vulnerabilidades-alvo nos workflows de teste) em tempo médio inferior a 2 horas por regra, incluindo estruturação de prompts, geração inicial, testes e refinamento. Esta métrica representa economia de tempo de pelo menos 75\% comparada à estimativa de desenvolvimento manual de 8 a 12 horas por regra, validando a contribuição metodológica do trabalho.

    \item \textbf{Cobertura Complementar Demonstrada:} A abordagem híbrida cobre significativamente mais classes de vulnerabilidades (mínimo de 80\%, equivalente a 5 de 6 classes) do que qualquer ferramenta isolada. Especificamente, deve-se demonstrar que o Agentic Radar cobre riscos não cobertos pelo Semgrep (ou vice-versa), validando a estratégia complementar.

    \item \textbf{Lacunas Documentadas com Rigor:} As limitações da abordagem proposta estão claramente identificadas, tecnicamente justificadas e acompanhadas de análise das causas fundamentais (limitações de análise de grafos, complexidade de análise semântica de expressões JavaScript, etc.). Oportunidades para trabalhos futuros são específicas e acionáveis, orientando pesquisas subsequentes.

    \item \textbf{Reprodutibilidade Assegurada:} A metodologia completa está documentada com detalhes suficientes para permitir que outros pesquisadores repliquem o estudo independentemente, incluindo versões de software utilizadas, comandos de instalação e configuração, estrutura dos workflows de teste, prompts exatos fornecidos ao Claude 3.5 Sonnet, e processo de validação de regras geradas.
\end{enumerate}

O não atingimento de qualquer um destes critérios constituiria uma lacuna fundamental que comprometeria a validade das conclusões do trabalho. Inversamente, a satisfação de todos os cinco critérios estabelece uma base sólida para afirmar que a abordagem híbrida proposta é viável, oferece valor demonstrável comparado a alternativas, e representa uma contribuição significativa para a área de segurança de plataformas Low-Code/No-Code.

% ==== Seção 4: Escopo da Solução ====
\section{Escopo da Solução e Requisitos}
\label{sec:escopo}

\textit{[Detalhar os casos de uso, arquitetura do sistema, requisitos técnicos e composição dos dados.]}

\subsection{Casos de Uso}

\textit{[Descrever os principais fluxos de interação com o sistema.]}

\subsubsection{Caso de Uso Principal}

\textit{[Detalhar o cenário principal de uso incluindo atores, pré-condições, fluxo principal e fluxos alternativos.]}

\textbf{Ator primário:} \textit{[Nome do ator]} \\
\textbf{Pré-condições:}
\begin{itemize}
    \item \textit{[Pré-condição 1]}
    \item \textit{[Pré-condição 2]}
\end{itemize}

\textbf{Fluxo principal:}
\begin{enumerate}
    \item \textit{[Passo 1]}
    \item \textit{[Passo 2]}
    \item \textit{[Passo 3]}
\end{enumerate}

\textbf{Fluxos alternativos:}
\begin{itemize}
    \item \textit{[Alternativa 1]}
    \item \textit{[Alternativa 2]}
\end{itemize}

\subsubsection{Casos Excluídos do Escopo}

\textit{[Listar explicitamente casos de uso que NÃO serão implementados na PoC.]}

\subsection{Arquitetura Proposta}

\textit{[Apresentar a arquitetura lógica do sistema, preferencialmente com diagramas, explicando as camadas, componentes e fluxo de dados. Incluir considerações de segurança (defense in depth, privacy by design).]}

% Exemplo de inclusão de figura (descomente quando tiver a imagem):
% \begin{figure}[htb]
% \centering
% \includegraphics[width=0.9\textwidth]{figures/arquitetura.png}
% \caption{Arquitetura lógica do sistema}
% \label{fig:arquitetura}
% \end{figure}

\subsubsection{Componentes Principais}

\textit{[Descrever em detalhes cada componente da arquitetura (frontend, backend, modelo de ML, sistema de logs, etc.) e suas responsabilidades.]}

\subsection{Requisitos Técnicos}

\textit{[Especificar as tecnologias, frameworks e infraestrutura necessários.]}

\subsubsection{Tecnologias e Frameworks}

\textit{[Listar em formato de tabela as tecnologias escolhidas para cada camada/componente com justificativas técnicas.]}

\begin{table}[htb]
\centering \small
\caption{Stack tecnológico}
\label{tab:stack}
\begin{tabular}{@{}llp{8cm}@{}}
\toprule
\textbf{Camada} & \textbf{Tecnologia} & \textbf{Justificativa} \\ \midrule
% Adicione suas tecnologias aqui
\bottomrule
\end{tabular}
\end{table}

\subsubsection{Infraestrutura Mínima}

\textit{[Especificar requisitos de hardware (CPU, GPU, RAM, disco) e rede para desenvolvimento, treinamento e deploy.]}

\subsubsection{Dados de Treinamento e Validação}

\textit{[Descrever a composição do dataset (quantidade, proporções, fontes), estratégia de divisão (treino/validação/teste) e considerações de privacidade.]}

\begin{table}[htb]
\centering \small
\caption{Composição do dataset}
\label{tab:dataset}
\begin{tabular}{@{}lrrr@{}}
\toprule
\textbf{Conjunto} & \textbf{Classe A} & \textbf{Classe B} & \textbf{Total} \\ \midrule
Treinamento (70\%) & & & \\
Validação (15\%) & & & \\
Teste (15\%) & & & \\ \midrule
\textbf{Total} & & & \\ \bottomrule
\end{tabular}
\end{table}

% ==== Seção 5: Threat Model ====
\section{Ameaças, Riscos e Controles}
\label{sec:threat-model}

\textit{[Aplicar metodologia STRIDE para identificação sistemática de ameaças, analisar riscos específicos de sistemas de ML, e estabelecer controles de mitigação.]}

\subsection{Superfícies de Ataque}

\textit{[Enumerar todos os pontos do sistema que podem ser explorados por atacantes.]}

\subsection{Análise STRIDE}

\textit{[Apresentar matriz completa de ameaças usando categorias STRIDE (Spoofing, Tampering, Repudiation, Information Disclosure, Denial of Service, Elevation of Privilege) com controles correspondentes.]}

\begin{table}[H]
\centering
\small
\caption{Matriz de ameaças STRIDE}
\label{tab:stride}
\begin{tabular}{@{}p{2cm}p{4.5cm}p{5cm}@{}}
\toprule
\textbf{Categoria} & \textbf{Ameaça} & \textbf{Controle} \\ \midrule
\textbf{Spoofing} & & \\
\textbf{Tampering} & & \\
\textbf{Repudiation} & & \\
\textbf{Information Disclosure} & & \\
\textbf{Denial of Service} & & \\
\textbf{Elevation of Privilege} & & \\
\bottomrule
\end{tabular}
\end{table}

\subsection{Ataques Específicos a Modelos de ML}

\textit{[Detalhar ameaças particulares a sistemas de aprendizado de máquina.]}

\subsubsection{Evasion Attacks}

\textit{[Descrever ataques adversariais que tentam enganar o modelo em tempo de inferência e controles de mitigação.]}

\subsubsection{Data Poisoning}

\textit{[Explicar riscos de contaminação do conjunto de treinamento e medidas preventivas.]}

\subsubsection{Model Extraction}

\textit{[Abordar riscos de roubo de modelo proprietário através de queries e contramedidas.]}

\subsection{Considerações de Privacidade (LINDDUN)}

\textit{[Aplicar metodologia LINDDUN para análise sistemática de riscos de privacidade (Linkability, Identifiability, Non-repudiation, Detectability, Disclosure, Unawareness, Non-compliance).]}

\subsection{Considerações Éticas}

\textit{[Discutir aspectos éticos do sistema.]}

\subsubsection{Vieses Algorítmicos}

\textit{[Identificar riscos de discriminação, estratégias de mitigação e métodos de avaliação de equidade.]}

\subsubsection{Transparência e Explicabilidade}

\textit{[Estabelecer compromissos de transparência e como a explicabilidade será implementada.]}

\subsubsection{Direito de Contestação}

\textit{[Definir processo para usuários contestarem decisões automatizadas.]}


% ==== Seção 7: Avaliação ====
\section{Avaliação Experimental}
\label{sec:avaliacao}

\textit{[Apresentar protocolo experimental, resultados obtidos, comparações e análises.]}

\subsection{Protocolo Experimental}

\textit{[Detalhar configuração de hardware/software e metodologia de avaliação.]}

\subsubsection{Configuração de Hardware e Software}

\textit{[Especificar ambiente de execução (GPU, RAM, OS, versões de bibliotecas).]}

\subsubsection{Repetição e Validação Cruzada}

\textit{[Explicar como a variância foi tratada (múltiplas execuções, seeds diferentes, intervalos de confiança).]}

\subsection{Resultados}

\textit{[Apresentar resultados quantitativos e qualitativos.]}

\subsubsection{Performance no Conjunto de Teste}

\textit{[Tabela com métricas principais (F1, Precisão, Recall, AUC-ROC) por modelo, comparando com metas estabelecidas.]}

\begin{table}[H]
\centering
\caption{Métricas principais por modelo}
\label{tab:resultados}
\begin{tabular}{@{}lcccc@{}}
\toprule
\textbf{Modelo} & \textbf{F1-Score} & \textbf{Precisão} & \textbf{Recall} & \textbf{AUC-ROC} \\ \midrule
% Adicione seus resultados aqui
\midrule
\textit{Meta Alvo} & & & & \\ \bottomrule
\end{tabular}
\end{table}

\subsubsection{Matriz de Confusão}

\textit{[Apresentar matriz de confusão com análise de falsos positivos e falsos negativos, identificando padrões nos erros.]}

\subsubsection{Performance por Tipo/Categoria}

\textit{[Detalhar desempenho em subcategorias do problema para identificar pontos fortes e fracos.]}

\subsubsection{Latência e Throughput}

\textit{[Reportar métricas de tempo de processamento e vazão, com breakdown por etapa do pipeline.]}

\subsection{Testes de Robustez}

\textit{[Avaliar resiliência do sistema.]}

\subsubsection{Transformações Geométricas e Degradações}

\textit{[Tabela mostrando degradação de performance sob transformações comuns (compressão, ruído, rotação).]}

\subsubsection{Ataques Adversariais}

\textit{[Resultados de testes contra ataques adversariais (FGSM, PGD) com diferentes intensidades.]}

\subsection{Análise de Vieses}

\textit{[Avaliar equidade do sistema.]}

\subsubsection{Equidade entre Subgrupos}

\textit{[Tabela comparando performance entre diferentes subgrupos para identificar possíveis discriminações.]}

\subsection{Explicabilidade: Análise Qualitativa}

\textit{[Avaliar qualidade das explicações geradas (Grad-CAM ou outra técnica) através de avaliação por especialistas ou métricas objetivas. Incluir exemplos visuais.]}

% ==== Seção 8: Discussão ====
\section{Discussão}
\label{sec:discussao}

\textit{[Interpretar resultados, discutir limitações, comparar com estado-da-arte e analisar implicações práticas.]}

\subsection{Alcance dos Objetivos}

\textit{[Avaliar se os objetivos estabelecidos foram atingidos, comparando resultados com metas.]}

\subsection{Limitações Identificadas}

\textit{[Discutir honestamente as limitações do trabalho.]}

\subsubsection{Limitações Técnicas}

\textit{[Identificar restrições técnicas (dependência de qualidade de entrada, escopo limitado, dataset sintético, etc.).]}

\subsubsection{Limitações de Segurança}

\textit{[Reconhecer vulnerabilidades residuais e aspectos de segurança que precisam ser melhorados para produção.]}

\subsection{Comparação com Estado-da-Arte}

\textit{[Tabela comparativa com trabalhos relacionados e sistemas comerciais, analisando posicionamento da PoC.]}

\subsection{Trade-offs Segurança vs. Usabilidade}

\textit{[Discutir compromissos feitos entre segurança, privacidade e usabilidade, justificando escolhas.]}

\subsection{Implicações Práticas}

\textit{[Analisar impactos práticos do trabalho.]}

\subsubsection{Para Usuários/Stakeholders}

\textit{[Discutir como o sistema pode ser utilizado na prática, benefícios e necessidades de treinamento.]}

\subsubsection{Para Políticas Públicas}

\textit{[Refletir sobre implicações para regulamentação, padronização e investimentos públicos.]}

% ==== Seção 9: Conclusões ====
\section{Conclusões e Próximos Passos}
\label{sec:conclusoes}

\textit{[Sintetizar contribuições, propor trabalhos futuros e fornecer recomendações.]}

\subsection{Síntese das Evidências}

\textit{[Resumir as principais conclusões demonstradas pela PoC de forma objetiva.]}

\subsection{Contribuições}

\textit{[Listar contribuições metodológicas, técnicas e científicas do trabalho.]}

\subsection{Trabalhos Futuros}

\textit{[Propor extensões e melhorias.]}

\subsubsection{Curto Prazo (3-6 meses)}

\textit{[Melhorias incrementais possíveis em 3-6 meses.]}

\subsubsection{Médio Prazo (6-12 meses)}

\textit{[Desenvolvimento de funcionalidades adicionais ou pilotos em 6-12 meses.]}

\subsubsection{Longo Prazo (1-2 anos)}

\textit{[Visão de evolução do sistema em 1-2 anos (escala, novas funcionalidades, federação).]}

\subsection{Recomendações}

\textit{[Fornecer orientações práticas para stakeholders interessados em adotar tecnologia similar.]}

% ==== Seção 10: Checklist de Conformidade ====
\section{Checklist de Conformidade (LGPD/Ética/Segurança)}
\label{sec:checklist}

\subsection{Lei Geral de Proteção de Dados (LGPD)}

\begin{itemize}[label=$\square$]
    \item \textbf{Minimização de dados (Art. 6º, III):}
    \begin{itemize}[label=$\square$]
        \item \textit{[Item de verificação 1]}
        \item \textit{[Item de verificação 2]}
    \end{itemize}

    \item \textbf{Finalidade específica (Art. 6º, I):}
    \begin{itemize}[label=$\square$]
        \item \textit{[Item de verificação 1]}
    \end{itemize}

    \item \textbf{Transparência (Art. 6º, VI):}
    \begin{itemize}[label=$\square$]
        \item \textit{[Item de verificação 1]}
    \end{itemize}

    \item \textbf{Segurança (Art. 46):}
    \begin{itemize}[label=$\square$]
        \item \textit{[Item de verificação 1]}
    \end{itemize}

    \item \textbf{Direito à explicação (Art. 20):}
    \begin{itemize}[label=$\square$]
        \item \textit{[Item de verificação 1]}
    \end{itemize}

    \item \textbf{Não discriminação (Art. 6º, IX):}
    \begin{itemize}[label=$\square$]
        \item \textit{[Item de verificação 1]}
    \end{itemize}
\end{itemize}

\subsection{Segurança da Informação}

\begin{itemize}[label=$\square$]
    \item \textbf{Criptografia:}
    \begin{itemize}[label=$\square$]
        \item \textit{[Item de verificação 1]}
    \end{itemize}

    \item \textbf{Autenticação e Autorização:}
    \begin{itemize}[label=$\square$]
        \item \textit{[Item de verificação 1]}
    \end{itemize}

    \item \textbf{Isolamento:}
    \begin{itemize}[label=$\square$]
        \item \textit{[Item de verificação 1]}
    \end{itemize}

    \item \textbf{Auditoria:}
    \begin{itemize}[label=$\square$]
        \item \textit{[Item de verificação 1]}
    \end{itemize}

    \item \textbf{Resiliência:}
    \begin{itemize}[label=$\square$]
        \item \textit{[Item de verificação 1]}
    \end{itemize}
\end{itemize}

\subsection{Ética e IA Responsável}

\begin{itemize}[label=$\square$]
    \item \textbf{Explicabilidade:}
    \begin{itemize}[label=$\square$]
        \item \textit{[Item de verificação 1]}
    \end{itemize}

    \item \textbf{Equidade:}
    \begin{itemize}[label=$\square$]
        \item \textit{[Item de verificação 1]}
    \end{itemize}

    \item \textbf{Accountability:}
    \begin{itemize}[label=$\square$]
        \item \textit{[Item de verificação 1]}
    \end{itemize}

    \item \textbf{Contestação:}
    \begin{itemize}[label=$\square$]
        \item \textit{[Item de verificação 1]}
    \end{itemize}

    \item \textbf{Transparência pública:}
    \begin{itemize}[label=$\square$]
        \item \textit{[Item de verificação 1]}
    \end{itemize}
\end{itemize}

\subsection{Ações Pendentes para Produção}

\begin{itemize}[label=$\square$]
    \item \textit{[Ação pendente 1]}
    \item \textit{[Ação pendente 2]}
\end{itemize}

% ==== Agradecimentos ====
\section*{Agradecimentos}

\textit{[Agradecer aos orientadores, colegas, instituições e fontes de financiamento quando aplicável.]}

% ==== Bibliografia ====
% \printbibliography

% ==== Apêndices ====
\appendix

\section{Detalhes de Implementação}
\label{app:implementacao}

\subsection{Estrutura de Diretórios}

\textit{[Apresentar árvore completa de diretórios do projeto.]}

\subsection{Comandos para Reprodução}

\textit{[Lista step-by-step de comandos para setup, treinamento, avaliação e deploy do sistema.]}

\section{Exemplos Adicionais de Resultados}
\label{app:exemplos}

\textit{[Figuras adicionais mostrando casos de sucesso e falha, exemplos visuais de classificações corretas e incorretas.]}

\section{Código-fonte Completo}
\label{app:codigo}

\textit{[Link para repositório público com código-fonte e licença de uso.]}

\end{document}
