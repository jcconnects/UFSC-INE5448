% !TeX program = lualatex
% This document requires LuaLaTeX for compilation

\documentclass[a4paper,12pt]{article}

% Engine detection and compatibility check
\RequirePackage{iftex}
\ifLuaTeX
  % Document is being compiled with LuaTeX - proceed normally
\else
  \PackageError{main}{This document requires LuaLaTeX for compilation.
    Please use: lualatex main.tex}
    {This document uses fontspec and other LuaTeX-specific features.}
\fi

\usepackage{fontspec}
\usepackage[brazil]{babel}

% Font configuration for LuaTeX
\setmainfont{Latin Modern Roman}
\setsansfont{Latin Modern Sans}
\setmonofont{Latin Modern Mono}

\usepackage{amsmath} % Para equações matemáticas
\usepackage{geometry}
\geometry{margin=1in}
\usepackage{hyperref}
\hypersetup{hidelinks}
\usepackage[sort]{natbib}
\usepackage{parskip}
\usepackage{longtable}
\usepackage{xcolor}
\usepackage{tabularx}
\usepackage[colorinlistoftodos,prependcaption,textsize=tiny]{todonotes}
\usepackage{comment}
\usepackage{graphicx} % Para resizebox
\usepackage{array} % Para colunas com largura fixa
\usepackage{listings}
\usepackage{siunitx} % For \si{\micro\second}
\usepackage[acronym,nonumberlist]{glossaries}
\usepackage{placeins}

% Configuring listings for code and log display
\lstset{
  basicstyle=\ttfamily\small,
  breaklines=true,
  breakatwhitespace=true,
  frame=single,
  numbers=left,
  numberstyle=\tiny,
  keywordstyle=\color{blue},
  commentstyle=\color{gray},
  showstringspaces=false,
  tabsize=2,
}


\title{Atividade 2 - Documentação Estruturada para
Projeto de IA e Segurança}
\author{João Pedro Schmidt Cordeiro - 22100628 \\ INE5448 - Tópicos Especiais em Aplicações Tecnológicas I}
\date{\today}

\begin{document}

\begin{titlepage}
    \centering
    \vspace*{\stretch{1.0}}
    {\Huge\bfseries Atividade 2 - Documentação Estruturada para Projeto de IA e Segurança\par}
    \vspace{\stretch{1.5}}
    {\Large João Pedro Schmidt Cordeiro - 22100628\par}
    {\large INE5448 - Tópicos Especiais em Aplicações Tecnológicas I\par}
    \vspace{\stretch{2.0}}
    {\today\par}
    \vfill
\end{titlepage}

\tableofcontents
\newpage

% --- PART 1: REQUIREMENTS DOCUMENTS ---
\section{Documentos de Requisitos}

\subsection{Product Requirements Document (PRD)}

\begin{description}
    \item[Título] PRD - Sistema de Detecção de Intrusões com IA (IDS-AI)
    \item[Versão] 1.0
    \item[Data] 28 de agosto de 2025
\end{description}

\subsubsection{Propósito}
O IDS-AI é um sistema de segurança de rede que utiliza inteligência artificial e aprendizado de máquina para monitorar, detectar e alertar sobre atividades maliciosas e anômalas em redes corporativas em tempo real. O objetivo é fornecer às equipes de segurança uma ferramenta proativa para identificar ameaças avançadas que contornam as defesas tradicionais.

\subsubsection{Público-alvo}
Empresas de médio e grande porte que necessitam de proteção robusta para sua infraestrutura de TI, especialmente aquelas que operam com dados sensíveis e estão sujeitas a regulamentações de segurança, como LGPD, GDPR, etc.

\subsubsection{Funcionalidades Principais}
\begin{enumerate}
    \item \textbf{Análise de Tráfego em Tempo Real:} Monitoramento contínuo de pacotes de dados na rede para identificar padrões suspeitos com baixa latência.
    \item \textbf{Detecção de Anomalias Baseada em ML:} Utilização de algoritmos de aprendizado de máquina não supervisionado para estabelecer uma linha de base ("baseline") do comportamento normal da rede e detectar desvios que possam indicar ataques \textit{zero-day}.
    \item \textbf{Alertas Inteligentes e Priorização:} Geração de alertas detalhados com classificação de severidade baseada no risco potencial, reduzindo a fadiga de alertas e permitindo que a equipe de segurança foque nas ameaças mais críticas.
    \item \textbf{Integração com Fontes de Threat Intelligence:} Enriquecimento das detecções com dados de fontes externas de inteligência de ameaças para identificar Indicadores de Comprometimento (IoCs) conhecidos, como IPs e domínios maliciosos.
    \item \textbf{Dashboard de Visualização e Relatórios:} Interface gráfica interativa que apresenta o estado da segurança da rede, detalhes dos alertas, e permite a geração de relatórios forenses para análise de incidentes.
\end{enumerate}

\subsubsection{Métricas de Sucesso}
\begin{itemize}
    \item \textbf{Taxa de Detecção de Ameaças Reais:} Percentual de ataques reais identificados com sucesso pelo sistema (> 95\%).
    \item \textbf{Taxa de Falsos Positivos:} Redução do número de alertas benignos classificados como maliciosos (< 2\% do total de alertas).
    \item \textbf{Mean Time to Detect (MTTD):} Tempo médio para detectar uma ameaça desde o seu início (meta: < 5 minutos).
    \item \textbf{Adoção do Produto:} Número de equipes de segurança que integram ativamente o IDS-AI em seu fluxo de trabalho diário.
\end{itemize}

\subsubsection{Considerações de Segurança Específicas}
\begin{itemize}
    \item \textbf{Privacidade de Dados:} O sistema deve ser capaz de anonimizar ou pseudoanonimizar dados sensíveis (payloads) para cumprir com a LGPD, sem comprometer a eficácia da detecção.
    \item \textbf{Segurança do Modelo de IA:} Implementar defesas contra ataques adversariais, como envenenamento de dados de treinamento e evasão de detecção.
    \item \textbf{Integridade dos Alertas:} Garantir que os alertas não possam ser manipulados ou suprimidos por agentes mal-intencionados.
\end{itemize}

\subsection{Technical Requirements Document (TRD)}
\begin{description}
    \item[Título] TRD - Sistema de Detecção de Intrusões com IA (IDS-AI)
    \item[Versão] 1.0
    \item[Data] 28 de agosto de 2025
\end{description}

\subsubsection{Arquitetura Geral}
O sistema será baseado em uma arquitetura de microsserviços para garantir escalabilidade e resiliência. Os componentes principais são:
\begin{itemize}
    \item \textbf{Coletores de Dados (Sensors):} Agentes leves implantados em pontos estratégicos da rede (SPAN ports, TAPs) para capturar o tráfego de rede.
    \item \textbf{Engine de Processamento de Dados:} Serviço responsável por normalizar, enriquecer e extrair features dos dados de rede brutos.
    \item \textbf{Servidor de Modelos de ML:} API que hospeda os modelos de detecção treinados e realiza inferências em tempo real.
    \item \textbf{Módulo de Alerta:} Serviço que recebe as detecções, as prioriza e as envia para os canais configurados (SIEM, e-mail, Slack).
    \item \textbf{Interface de Usuário (Dashboard):} Aplicação web para visualização dos dados e gerenciamento do sistema.
\end{itemize}

\subsubsection{Requisitos de IA}
\begin{itemize}
    \item \textbf{Algoritmos de Detecção de Anomalias:} Utilização de um ensemble de algoritmos, incluindo \textbf{Isolation Forest} (para eficiência) e \textbf{Autoencoders} (para detecções complexas).
    \item \textbf{Algoritmos para Tráfego Sequencial:} Uso de Redes Neurais Recorrentes (\textbf{LSTM}) para analisar sequências de pacotes e identificar ataques multi-etapa.
    \item \textbf{Frameworks:} Python com Scikit-learn, TensorFlow/PyTorch.
    \item \textbf{Retreinamento:} O pipeline de MLOps deve suportar o retreinamento automático dos modelos a cada 30 dias para se adaptar a mudanças no comportamento da rede.
\end{itemize}

\subsubsection{Requisitos de Segurança}
\begin{itemize}
    \item \textbf{Autenticação e Autorização:} Acesso ao dashboard e APIs protegido por OAuth 2.0 com Role-Based Access Control (RBAC).
    \item \textbf{Criptografia:} Comunicação entre microsserviços e com sistemas externos via mTLS. Dados em repouso (logs, modelos) criptografados com AES-256.
    \item \textbf{Logs de Auditoria:} Todas as ações administrativas e acessos ao sistema devem ser registrados em logs imutáveis.
\end{itemize}

\subsubsection{Requisitos de Performance}
\begin{itemize}
    \item \textbf{Latência de Detecção:} O tempo entre a captura de um pacote e a geração de um alerta deve ser inferior a 200 milissegundos.
    \item \textbf{Throughput:} Os coletores devem suportar uma taxa de captura de até 10 Gbps por sensor.
    \item \textbf{Escalabilidade:} A arquitetura deve suportar escalonamento horizontal para lidar com o aumento do volume de tráfego de rede.
\end{itemize}

\subsubsection{Integrações Necessárias}
\begin{itemize}
    \item \textbf{SIEM:} Envio de alertas em formatos padronizados (CEF, LEEF) para plataformas como Splunk, QRadar e ArcSight.
    \item \textbf{Firewalls/SOAR:} Integração via API para permitir ações de resposta automatizadas, como o bloqueio de um endereço IP malicioso.
    \item \textbf{Bases de Dados de Threat Intelligence:} Conexão com APIs de feeds de inteligência (ex: VirusTotal, AbuseIPDB) via protocolo STIX/TAXII.
\end{itemize}

\newpage

% --- PART 2: TECHNICAL DECISION ---
\section{Documentação de Decisões Técnicas}

\subsection{Architectural Decision Record (ADR-001)}

\begin{description}
    \item[Título] ADR-001: Seleção de Algoritmo para Detecção de Anomalias
    \item[Status] Aceito
    \item[Data] 28 de agosto de 2025
\end{description}

\subsubsection{Contexto}
A funcionalidade central do IDS-AI é a detecção de anomalias em tempo real. A escolha do algoritmo de Machine Learning inicial é uma decisão crítica que impactará diretamente a precisão, a performance, a complexidade de implementação e a manutenibilidade do sistema. Precisamos de um modelo que seja eficaz na identificação de desvios do comportamento normal da rede, computacionalmente eficiente para operar em tempo real e relativamente simples de implementar na primeira versão do produto (MVP).

\subsubsection{Opções Consideradas}
\begin{enumerate}
    \item \textbf{Isolation Forest:}
    \begin{itemize}
        \item \textit{Descrição:} Um algoritmo de ensemble não supervisionado que isola anomalias em vez de criar um perfil de dados normais. É eficiente em datasets de alta dimensão.
        \item \textit{Prós:} Muito rápido para treinar e prever, baixo consumo de memória, bom desempenho em diversos cenários.
        \item \textit{Contras:} Menos eficaz na detecção de anomalias contextuais (que dependem de uma sequência de eventos).
    \end{itemize}

    \item \textbf{LSTM (Long Short-Term Memory) Autoencoder:}
    \begin{itemize}
        \item \textit{Descrição:} Uma rede neural recorrente usada para aprender o padrão sequencial de dados normais. Anomalias são detectadas quando a reconstrução de uma nova sequência tem um erro alto.
        \item \textit{Prós:} Excelente para dados temporais (fluxos de rede), capaz de capturar padrões complexos e ataques multi-etapa.
        \item \textit{Contras:} Computacionalmente caro, requer mais dados para treinamento, mais complexo de implementar e otimizar.
    \end{itemize}

    \item \textbf{One-Class SVM (Support Vector Machine):}
    \begin{itemize}
        \item \textit{Descrição:} Um algoritmo que aprende uma fronteira ao redor dos dados normais. Qualquer ponto que caia fora dessa fronteira é considerado uma anomalia.
        \item \textit{Prós:} Bem estabelecido e com base teórica sólida.
        \item \textit{Contras:} Não escala bem com o número de amostras, sensível a hiperparâmetros, pode ter dificuldade com estruturas de dados complexas.
    \end{itemize}
\end{enumerate}

\subsubsection{Decisão}
A decisão é adotar o \textbf{Isolation Forest} como o principal algoritmo para a detecção de anomalias no MVP do IDS-AI.
\\ \\
\textbf{Justificativa:} O Isolation Forest oferece o melhor equilíbrio entre performance e precisão para uma primeira versão. Sua alta velocidade de inferência é crucial para atender ao requisito de detecção em tempo real. A simplicidade de implementação permitirá uma entrega mais rápida de valor, validando a arquitetura central do produto. O LSTM Autoencoder será considerado para uma versão futura (v2.0) para adicionar uma camada de detecção de ameaças temporais mais sofisticada.

\subsubsection{Consequências}
\begin{itemize}
    \item \textbf{Positivas:}
    \begin{itemize}
        \item Rápido desenvolvimento e implantação do MVP.
        \item Baixa sobrecarga computacional nos servidores de inferência.
        \item Estabelece uma base sólida de detecção que pode ser facilmente expandida.
    \end{itemize}
    \item \textbf{Negativas:}
    \begin{itemize}
        \item O sistema pode não detectar ataques sutis que se desenrolam ao longo do tempo e dependem da ordem dos eventos.
        \item A interpretabilidade dos resultados do Isolation Forest é limitada, o que pode dificultar a análise de causa raiz de alguns alertas.
        \item Haverá a necessidade de alocar recursos no futuro para pesquisar e implementar a solução baseada em LSTM.
    \end{itemize}
\end{itemize}

\newpage

% --- PART 3: AI EXPERIMENTATION ---
\section{Experimentação com IA}

Para a elaboração desta atividade, diversas ferramentas de IA foram empregadas como assistentes de desenvolvimento e pesquisa, incluindo Gemini 2.5 Pro~\cite{gemini25}, ChatGPT 5~\cite{chatgpt5}, NotebookLM~\cite{notebooklm_ref} e o ambiente de desenvolvimento Cursor~\cite{cursor}.

\subsection{Engenharia de Prompts para Documentação}

\subsubsection{Prompt Estruturado Utilizado}
O prompt a seguir foi fornecido a uma ferramenta de IA (GPT-4\cite{openai_gpt4}) para gerar a seção "Procedimentos de Resposta a Incidentes" do TRD. O procedimento segue as 6 fases do ciclo de vida de resposta a incidentes do NIST\cite{nist_sp800_61r2}.

\begin{lstlisting}[caption=Prompt para Geração de Seção do TRD]
# Persona
Aja como um Engenheiro de Segurança Sênior especializado em resposta a incidentes e automação (SOAR).

# Contexto
Estou criando o Documento de Requisitos Técnicos (TRD) para um novo Sistema de Detecção de Intrusões baseado em IA (IDS-AI). O sistema monitora o tráfego de rede em tempo real, detecta anomalias usando Machine Learning e se integra a sistemas SIEM e firewalls via API. O público-alvo são equipes de segurança de médias e grandes empresas.

# Tarefa
Gere uma seção detalhada para o TRD chamada "Procedimentos de Resposta a Incidentes". A seção deve descrever um plano de resposta padrão que possa ser iniciado a partir de um alerta crítico gerado pelo IDS-AI.

# Requisitos do Conteúdo
O procedimento deve seguir as 6 fases do ciclo de vida de resposta a incidentes do NIST:
1.  **Preparação:** O que a equipe precisa ter em vigor ANTES de um incidente.
2.  **Identificação:** Como confirmar e analisar um alerta do IDS-AI.
3.  **Contenção:** Ações imediatas para limitar o dano.
4.  **Erradicação:** Como remover a causa raiz da ameaça.
5.  **Recuperação:** Passos para restaurar os sistemas à operação normal.
6.  **Lições Aprendidas (Post-Incident):** Como melhorar o processo após o incidente.

# Formato
Use títulos claros para cada fase. Dentro de cada fase, utilize bullet points para descrever as ações específicas. Seja objetivo e técnico.
\end{lstlisting}

\subsubsection{Resposta Gerada pela IA}
\begin{center}
    \textbf{--- Início da Resposta Gerada ---}
\end{center}

\subsubsection*{Procedimentos de Resposta a Incidentes}
Esta seção descreve o fluxo de trabalho padrão para responder a um alerta de alta severidade gerado pelo IDS-AI, alinhado com as melhores práticas do setor.

\begin{description}
    \item[1. Preparação]
    \begin{itemize}
        \item Manter uma lista de contatos atualizada da equipe de resposta a incidentes (CSIRT).
        \item Garantir que os analistas de segurança tenham acesso e permissões necessárias nas ferramentas integradas (SIEM, Firewall, EDR).
        \item Desenvolver e manter playbooks de resposta para os tipos mais comuns de alertas (ex: detecção de malware, exfiltração de dados, varredura de rede).
        \item Realizar treinamentos e simulações de incidentes periodicamente.
    \end{itemize}

    \item[2. Identificação]
    \begin{itemize}
        \item Ao receber um alerta crítico do IDS-AI, o analista de plantão (on-call) inicia a triagem em até 15 minutos.
        \item Correlacionar o alerta com outras fontes de dados no SIEM para obter contexto (logs de firewall, autenticação, etc.).
        \item Analisar os detalhes do alerta no dashboard do IDS-AI: pacotes capturados, IPs de origem/destino, e a justificativa da anomalia pelo modelo de ML.
        \item Validar se o alerta é um verdadeiro positivo. Se confirmado, declarar um incidente formal e escalar para a equipe sênior.
    \end{itemize}
    
    \item[3. Contenção]
    \begin{itemize}
        \item Isolar os ativos afetados da rede para prevenir a propagação da ameaça.
        \item Utilizar a integração do IDS-AI com o firewall para bloquear os endereços IP de origem maliciosos via API.
        \item Desativar contas de usuário comprometidas, se aplicável.
        \item Preservar evidências voláteis (memória RAM) e não voláteis (discos) dos sistemas afetados para análise forense.
    \end{itemize}

    \item[4. Erradicação]
    \begin{itemize}
        \item Identificar e remover a causa raiz do incidente (ex: remover o malware, corrigir a vulnerabilidade explorada).
        \item Realizar uma varredura completa nos sistemas afetados para garantir que não há persistência do atacante.
        \item Aplicar patches de segurança e fortalecer as configurações dos sistemas para prevenir a reinfecção.
    \end{itemize}
    
    \item[5. Recuperação]
    \begin{itemize}
        \item Restaurar os sistemas a partir de backups limpos e verificados.
        \item Monitorar de perto os sistemas restaurados para garantir que estão operando normalmente e que a ameaça não retornou.
        \item Reintroduzir gradualmente os sistemas na rede de produção após validação.
    \end{itemize}

    \item[6. Lições Aprendidas (Post-Incident)]
    \begin{itemize}
        \item Conduzir uma reunião de post-mortem em até 5 dias úteis após a resolução do incidente.
        \item Documentar a linha do tempo do incidente, o impacto, as ações tomadas e as lições aprendidas.
        \item Criar itens de ação para melhorar as defesas, ferramentas e procedimentos (ex: ajustar a sensibilidade do modelo de ML, criar uma nova regra de detecção no SIEM).
        \item Atualizar os playbooks de resposta a incidentes com base nos aprendizados.
    \end{itemize}
\end{description}
\begin{center}
    \textbf{--- Fim da Resposta Gerada ---}
\end{center}

\subsubsection{Análise Crítica}
\begin{itemize}
    \item \textbf{O que funcionou bem:} A IA compreendeu perfeitamente a estrutura solicitada (as 6 fases do NIST) e a persona de um engenheiro de segurança. O conteúdo gerado é tecnicamente sólido, relevante e bem organizado. A linguagem é apropriada para um documento técnico. A IA também conseguiu conectar as ações ao contexto do IDS-AI (ex: "analisar os detalhes do alerta no dashboard do IDS-AI").
    
    \item \textbf{O que precisou ser ajustado:} A resposta é genérica por natureza, o que é esperado. Para um documento real, seria necessário adicionar detalhes específicos da organização, como nomes de ferramentas (ex: "bloquear IPs no Firewall \textit{Palo Alto}"), nomes de equipes e SLAs (Service Level Agreements) exatos (ex: "iniciar triagem em \textit{exatos} 15 minutos"). A resposta é um excelente \textit{template}, mas a customização humana é indispensável para torná-la acionável.
\end{itemize}

\subsection{IA como Base de Conhecimento}

\subsubsection{Prompt Hipotético para IA}
Um desenvolvedor, de posse do PRD e do TRD criados, poderia usar o seguinte prompt em uma ferramenta de IA avançada (como o GitHub Copilot Enterprise, que pode ser contextualizado com a base de código e documentação do projeto) para acelerar a implementação.

\begin{lstlisting}[caption=Prompt Hipotético de Implementação]
# Contexto
Estou trabalhando no projeto IDS-AI. Tenha como base o PRD e o TRD fornecidos.

# Tarefa
Gere um esqueleto de código em Python para a funcionalidade "Alertas Inteligentes e Priorização".

# Requisitos do Código
1. Crie uma classe `AlertManager`.
2. O construtor deve receber a URL da API do SIEM.
3. Implemente um método `process_detection(detection_data)` que recebe um dicionário com dados da detecção (ex: `{'source_ip': '...', 'threat_type': '...', 'confidence_score': 0.85}`).
4. Dentro deste método, implemente a lógica de priorização descrita no TRD (considere o `confidence_score` e o tipo de ameaça). A prioridade deve ser "Baixa", "Média", "Alta" ou "Crítica".
5. Crie um método privado `_send_to_siem(alert_payload)` que formate o alerta para o padrão CEF (Common Event Format) e o envie para a API do SIEM usando uma requisição POST.
6. Adicione docstrings e type hints ao código.
\end{lstlisting}

\subsubsection{Como a Documentação Facilitaria a Resposta da IA}
A documentação estruturada seria fundamental para que a IA gerasse uma resposta precisa e útil.

\begin{enumerate}
    \item \textbf{O PRD forneceria o "porquê":} A seção \textit{"Funcionalidades Principais"} do PRD confirmaria para a IA a importância estratégica dos "Alertas Inteligentes e Priorização", garantindo que a solução gerada esteja alinhada com os objetivos do produto.
    
    \item \textbf{O TRD forneceria o "como":}
    \begin{itemize}
        \item A seção \textit{"Arquitetura Geral"} informaria à IA que existe um "Módulo de Alerta" e uma "Interface de Usuário", ajudando-a a entender onde a classe `AlertManager` se encaixaria.
        \item A seção \textit{"Integrações Necessárias"} seria a mais crítica. Ela especifica a necessidade de integração com \textbf{SIEM} e o formato \textbf{CEF}. Sem essa informação, a IA poderia gerar um código genérico de log ou usar um formato incorreto.
        \item A seção \textit{"Requisitos de IA"} mencionaria os modelos e seus outputs (como um score de confiança), o que daria à IA o contexto necessário para implementar a lógica de priorização solicitada no prompt.
    \end{itemize}
\end{enumerate}

Em resumo, sem a documentação, a IA teria que fazer suposições. Com o PRD e o TRD, a IA age como um desenvolvedor júnior que recebeu especificações claras de um sênior, resultando em um código muito mais alinhado com as necessidades reais do projeto, economizando tempo de desenvolvimento e refatoração.


\bibliographystyle{plain}
\bibliography{references}

\end{document}